\section{Training and Evaluation Datasets}
\subsection{Training Datasets}
\label{secs:data}

We train on a combination of image-text datasets for all \bdraw models. The data includes the publicly available LAION-400M dataset~\cite{schuhmann2021laion}; \webfit, a filtered subset of the full 1.8 billion examples used to train the ALIGN model~\cite{align-paper}; JFT-4B dataset~\cite{zhai2021scaling}, which has images with text annotation labels. For textual descriptions of JFT, we randomly switch between the original labels as text (concatenated if an image has multiple labels) or machine-generated captions from a SimVLM model~\cite{wang2021simvlm}. We discuss the limitations of the data in Section~\ref{secs:broader}. For all image inputs, we follow the DALL-E dVAE input processing (Section A.2. Training in~\cite{ramesh2021zero}) for image tokenizer training and the DALL-E Transformer input processing (Section B.2. Training in~\cite{ramesh2021zero}) for encoder-decoder training.

\subsection{Evaluation Datasets}
\begin{table*}[t]
\begin{center}
\footnotesize
\setlength\tabcolsep{2pt}
\resizebox{\textwidth}{!}{
\begin{tabular}{lccccccp{4.5cm}ll}
\toprule
\normalsize Dataset && \normalsize Train & \normalsize Val & & \normalsize AvgWords && \normalsize \centering Caption  && \normalsize Image   \\
\midrule
MS-COCO (2014)~\cite{lin2014microsoft} && 82K & 40K &&  10.5 && \textit{``A bowl of broccoli and apples with a utensil.''} &&   \multirow{2}{*}{\includegraphics[height=1.65in,width=1.65in]{figures/COCO_val2014_000000551660.jpg}} \\[0.35cm]
Localized Narratives\\ (COCO subset) ~\cite{PontTuset_eccv2020} && 134K & 8K  && 42.1 && \textit{``In this picture, we see a bowl containing the chopped apples and broccoli. In the background, we see a white table on which seeds or grains, broccoli, piece of fruit, water glass and plates are placed. This table is covered with a white and blue color cloth. This picture is blurred in the background.''} &&  \\
\bottomrule
\end{tabular}
}
\caption{Evaluation data statistics and examples. Images from the COCO portion of Localized Narratives come from the MS-COCO (2017) set; Localized Narratives descriptions are four times the length of captions in MS-COCO on average. The example above highlights the massive difference in detail between MS-COCO and Localized Narratives for the \textit{same} image.}
\label{tabs:datasets}
\end{center}
\end{table*}

We evaluate our models on MS-COCO (2014) \cite{lin2014microsoft} and Localized Narratives \cite{PontTuset_eccv2020}, summarized in Table~\ref{tabs:datasets}. MS-COCO is the current standard dataset for measuring both zero-shot and finetuned text-to-image generation performance, which makes it a consistent point of comparison with prior work. However, MS-COCO captions are short, high-level characterizations of their corresponding images. For a more comprehensive evaluation, we also use the COCO portion of Localized Narratives (LN-COCO), which provides longer, detailed descriptions of images corresponding to the MS-COCO (2017) dataset, and compare \bdraw's performance on LN-COCO against \cite{trecs2020, zhang2021cross}. These long-form descriptions are typically quite different from the descriptions used to train large text-to-image generation models. This provides a measure of generalization to out-of-domain distributions, as well as the finetuning capability of these models. Regardless of the community's current focus on zero-shot performance, the ability to finetune effectively is also important for adapting an open-domain text-to-image generation model to work with a specific application area or domain.

\subsection{\bcp{}}
\label{secs:bcp}

\begin{table*}
\centering
\renewcommand{\arraystretch}{1.5}
{\footnotesize
\begin{tabular}{p{1.5cm}p{11.5cm}}
\toprule
Category & \multicolumn{1}{c}{Examples} \\
\midrule
\textsc{Abstract} & \textit{commonsense; happiness; hope; insight; 300; 101; golden ratio; Fibonacci number} \\
\textsc{Animals} & \textit{a Tyrannosaurus Rex; 7 dogs sitting around a poker table, two of which are turning away; a white rabbit in blue jogging clothes doubled over in pain while a turtle wearing a red tank top dashes confidently through the finish line} \\
\textsc{Arts} & \textit{a dutch baroque painting of a horse in a field of flowers; a painting of a fox in the style of starry night; A raccoon wearing formal clothes, wearing a top hat and holding a cane. The raccoon is holding a garbage bag. Oil painting in the style of Vincent Van Gogh.} \\
\textsc{Illustra- tions} & \textit{concurrent lines; a diagram of brain function; concentric squares fading from yellow on the outside to deep orange on the inside; a drawing of a series of musical notes wrapped around the Earth} \\
\textsc{Outdoor Scenes} & \textit{a grand piano next to the net of a tennis court; a white country home with a wrap-around porch; beautiful fireworks in the sky with red, white and blue; a peaceful lakeside landscape with migrating herd of sauropods} \\
\textsc{People} & \textit{the Beatles crossing Abbey road; a woman with long hair next to a luminescent bird; a tall man stooping down to enter a low red sports car; a politician wearing a soccer jersey and holding a volleyball while giving a speech on a stage} \\
\textsc{Vehicles} & \textit{an F1; a boat in the canals of venice; a car with tires that have yellow rims; a blue semi-truck and its trailer jumping over a row of motorcycles. there are metal ramps on either side of the motorcycles.} \\
\textsc{World Knowledge} & \textit{U.S. 101; the skyline of New York City; An aerial view of Ha Long Bay without any boats; view of the Great Wall from its base; A close-up high-contrast photo of Sydney Opera House sitting next to Eiffel tower, under a blue night sky of roiling energy, exploding yellow stars, and radiating swirls of blue.} \\
\bottomrule
\end{tabular}
}
\caption{Sample categories in the \bcp{} (\bcpa{}) benchmark. Examples (separated by semicolons) range from abstract concepts such as ``golden ratio'' to concrete ones such as ``the skyline of New York City''. For full descriptions of all categories, see Appendix \ref{secs:appendix_bcp}.}
\label{t:bcp_categories}
\end{table*}


Existing benchmarks like MS-COCO \cite{lin2014microsoft} and Localized Narratives \cite{PontTuset_eccv2020} are clearly useful for measuring the progress of text-to-image synthesis systems, but the descriptions available in them are generally limited to everyday scenes and objects found in natural images. This limits their representation of a broad spectrum of prompts -- in particular, they lack prompts that allow us to better probe model capabilities on open-domain text-to-image generation. For example, MS-COCO captions are brief characterizations of high level participants and actions in images; these typically cover common scenarios and are oriented toward objects. Localized Narratives has highly-detailed descriptions, but also emphasizes natural scenes and objects. Recently, the work by \cite{park2021benchmark} focuses on the text-to-image generation task, but is limited to only two scenarios, unseen object-color (\eg, ``blue petal'') and object-shape (\eg, ``long beak'').
Motivated by these shortcomings, we present \bcp{} (\bcpa{}), a set of \bcpsize{} diverse English prompts that allow us to more comprehensively evaluate and test the limits of text-to-image synthesis models.

\begin{table*}
\centering
\renewcommand{\arraystretch}{1.5}
{\footnotesize
\begin{tabular}{p{1.8cm}p{11.2cm}}
\toprule
Challenge & \multicolumn{1}{c}{Examples} \\
\midrule
\textsc{Basic} & \textit{a rabbit; U.S. 101; a margarita; lily pads; brain coral; The Alamo; a family} \\
\textsc{Complex} & \textit{the Sydney Opera House with the Eiffel tower sitting on the right, and Mount Everest rising above; a white rabbit in blue jogging clothes doubled over in pain while a turtle wearing a red tank top dashes confidently through the finish line; Oil-on-canvas painting of a blue night sky with roiling energy. A fuzzy and bright yellow crescent moon shining at the top. Below the exploding yellow stars and radiating swirls of blue, a distant village sits quietly on the right. Connecting earth and sky is a flame-like cypress tree with curling and swaying branches on the left. A church spire rises as a beacon over rolling blue hills.} \\
\textsc{Imagination} & \textit{a four-eyed horse; a toaster shaking hands with a microwave; a peaceful lakeside landscape with migrating herd of sauropods; a flower with large red petals growing on the moon's surface; A rusty spaceship blasts off in the foreground. A city with tall skyscrapers is in the distance, with a mountain and ocean in the background. A dark moon is in the sky. realistic high-contrast anime illustration.} \\
\textsc{Linguistic Structures} & \textit{Incomprehensibilities; Pneumonoultramicroscopicsilicovolcanoconiosis; The horse raced past the barn fell; One morning I chased an elephant in my pajamas; The dog chased the cat, which ran up a tree. It waited at the top; The dog chased the cat, which ran up a tree. It waited at the bottom.} \\
\textsc{Perspective} & \textit{the back of a violin; an extreme close-up view of a capybara sitting in a field; tall buildings seen through a window with rain on it; view from below of a tall white ladder with just one rung leaning up against a yellow brick wall} \\
\textsc{Quantity} & \textit{an owl family; 7 dogs sitting around a poker table, two of which are turning away; a basketball game between a team of four cats and a team of three dogs} \\
\textsc{Writing~\& Symbols} & \textit{a grumpy porcupine handing a check for \$10,000 to a smiling peacock; a group of cats in a meeting. there is a whiteboard with "stack more layers" written on it; The saying "BE EXCELLENT TO EACH OTHER" written on a red brick wall with a graffiti image of a green alien wearing a tuxedo. A yellow fire hydrant is on a sidewalk in the foreground.} \\
\bottomrule
\end{tabular}
}
\caption{Sample challenge aspects in the \bcpa{} benchmark. Examples (separated by semicolons) range from basic to complex ones such as a full description of the Starry Night ``Oil-on-canvas painting of a blue night sky ... rolling blue hills''. For full descriptions of all challenge aspects, see Appendix \ref{secs:appendix_bcp}.}
\label{t:bcp_trickiness}
\end{table*}


Each prompt in the \bcpa{} benchmark is associated with two labels: (1) {\it Category}, indicating a broad group that a prompt belongs to, and (2) {\it Challenge}, highlighting an aspect which makes a prompt difficult.
Table~\ref{t:bcp_categories} provides a few samples of categories (out of \bcpcat{} options) used in \bcpa{}, ranging from abstract concepts such as the ``golden ratio'' to concrete world-knowledge ones such as ``the skyline of New York City''. Similarly, Table~\ref{t:bcp_trickiness} lists a sample of challenge aspects (out of \bcptrick{}), ranging from basic ones such as ``a rabbit'' to complex ones such as a full description of the painting \textit{Starry Night} (``Oil-on-canvas painting of a blue night sky ... A church rises as a beacon against rolling blue hills.''). For example, the prompt ``a peaceful lakeside landscape with migrating herd of sauropods'' is categorized as \bcpstyle{Outdoor Scenes}, while its challenge aspect is \bcpstyle{Imagination}. Similarly, the prompt ``7 dogs sitting around a poker table, two of which are turning away'' has \bcpstyle{Animals} as category and \bcpstyle{Quantity} as challenge aspect. These two views of a prompt allows us to analyze a model's capabilities from both aspects--the overall content generated and the subtle details captured.

\begin{figure}[tbh!]
    \centering
    \includegraphics[width=1\textwidth]{figures/bcp.png}
    \caption{A summary of the \bcp{} (\bcpa{}) set of \bcpsize{} descriptions, spanning across many {\it category labels} (left) and {\it challenge aspects} (right).
    }
    \label{figs:bcp}
\end{figure}

We created \bcp{}
both by thinking of novel prompts and by manually curating and sampling prompts from recent papers~\cite{ramesh2021zero, ding2021cogview, vqdiffusion, nichol2021glide, ramesh2022hierarchical} (which accounts for about 7\% of prompts in \bcpa{}). While it is possible to assign multiple categories and challenge aspects to a prompt, we chose to reduce the complexity of model analysis by manually deciding on a single primary category and challenge aspect for each prompt. For example, when there is a proper noun, we will prefer to label the prompt as \bcpstyle{World Knowledge} (e.g., ``a painting of street in Paris'') over other categories, e.g., \bcpstyle{Arts}. We also prioritize categories that have fewer examples, \eg, \bcpstyle{Arts}, over those with plenty, such as \bcpstyle{Animals}, as in the case of the prompt ``A raccoon wearing formal clothes, wearing a tophat and holding a cane. The raccoon is holding a garbage bag. Oil painting in the style of Vincent Van Gogh.'' The \bcpstyle{People} category always takes priority, \eg, ``a team playing baseball at the beach'' is labeled as \bcpstyle{People} instead of \bcpstyle{Outdoor Scenes}; we do this to ease future work on fairness and bias interested in using \bcp, as it can be easily split to include or exclude prompts involving people.
Figure~\ref{figs:bcp} highlights the distribution of category labels and challenge aspects over the \bcpsize{} prompts. One can group these prompts into levels of difficulties by the challenge aspects: {\it Standard} includes \bcpstyle{Basic} and \bcpstyle{Simple Detail} (about 1/3 of the prompts); {\it Intermediate} includes \bcpstyle{Fine-grained Detail} and \bcpstyle{Style \& Format} (also about 1/3 of the prompts); and {\it Challenging} includes the remaining 7 challenge aspects such as \bcpstyle{Imagination}, \bcpstyle{Quantity}, \bcpstyle{Complex}, and \bcpstyle{Linguistic Structures}.

It is also worth mentioning DrawBench, a concurrently developed benchmark of 200 prompts introduced in \cite{imagen}. It has eleven labels that mix both categories (\eg, ``DALL-E'') and challenging aspects (\eg, ``Counting''). \bcp{}, in contrast, teases apart these two dimensions, with \bcpcat{} categories and \bcptrick{} challenging aspects, allowing for richer categorizations of prompts and more fined-grained analyses, together with \(8\times\) more prompts. Both benchmarks contain prompts that present strong challenges for current best models--including DALL-E 2, Imagen and \bdraw--and hopefully will inspire further benchmarks to increase the level of difficulty as future models continually improve.
